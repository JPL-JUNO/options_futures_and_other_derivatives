\documentclass{article}
\usepackage{ctex}
\usepackage{amsmath, amssymb}
\begin{document}
期权价格条件(以无套利市场中无分红的资产的期权为例)
\begin{enumerate}
    \item $0\leq C_E\leq C_A \leq S$, $0\leq P_E\leq P_A \leq K$,即欧式小于美式,看涨小于标的资产价格,看跌小于执行价;
    \item $C_E\geq \max(S_0-Ke^{-rt}, 0)$, $P_E\geq \max(Ke^{-rt}-S_0, 0)$
    \item 其他条件相同,执行不同的期权有,若 $K_1\leq K_2$,则$C_E(K_1) \geq C_E(K_2)$, $P_E(K_1)\geq P_E(K_2)$;
    \item 其他条件相同,到期日不同的美式期权,若 $t_1\leq t_2$,则 $C_A(t_1)\leq C_A(t_2)$, $P_A(t_1)\leq P_A(t_2)$,欧式期权不适用;
    \item 欧式期权看涨看跌平价关系,$C_E+Ke^{-rt}=P_E+S_0$,推广到美式期权,$C_A+Ke^{-rt}\leq P_A+S_0\leq C_A+K$
    \item 其他条件相同,执行价不同的欧式看涨期权,若 $K_1<K_2$,则 
    \begin{equation}
        \begin{aligned}
            C_E(K_1)-C_E(K_2)&\leq (K_2-K_1)e^{-rt}\\
            P_E(K_1)-P_E(K_2)&\leq (K_2-K_1)e^{-rt}\\
            C_E(K_1)-C_E(K_2)+P_E(K_1)-P_E(K_2)&= (K_2-K_1)e^{-rt}\\
        \end{aligned}
    \end{equation}
    \item 其他条件相同,执行价不同的欧式期权,若 $K_1<K_2<K_3$,令 $\lambda=\frac{K_3-K_2}{K_3-K_1}$,则
    \begin{equation}
        \begin{aligned}
            C_E(K_2)&\leq \lambda C_E(K_1)+(1-\lambda)C_E(K_3)\\
            P_E(K_2)&\leq \lambda P_E(K_1)+(1-\lambda)P_E(K_3)\\
        \end{aligned}
    \end{equation}
\end{enumerate}
期权模型,单步二叉树模型
\begin{equation}
    C=e^{-rt}[pC_u+(1-p)C_u]
\end{equation}
其中,$p$ 也称为风险中性概率,计算方法如下:
$$p=\frac{e^{rt}-d}{u-d}$$
期权模型,B-S-M模型
无红利标的的资产欧式看涨期权 C(看跌期权 P)的定价公式为:
\begin{equation}
    \begin{aligned}
        C&=SN(d_1)-Ke^{-rt}N(d_2)\\
        P&=Ke^{-rt}N(-d_2)-SN(-d_1)\\
    \end{aligned}
\end{equation}
其中,
\begin{equation}
    \begin{aligned}
        d_1=\frac{\ln(S/k)+[r+(\sigma^2/2)]T}{\sigma \sqrt{T}}\\
        d_2=\frac{\ln(S/k)+[r-(\sigma^2/2)]T}{\sigma \sqrt{T}}\\
    \end{aligned}
\end{equation}
注意事项:
\begin{enumerate}
    \item 从公式可以看出,在风险中性的前提下,投资者的预期收益率 $\mu$ 用无风险收益率 $r$ 替代
    \item $N(d_2)$ 表示在无风险中性市场中 $S_T$(标的资产在时刻的价格)大于 $K$ 的概率,或者说是欧式看涨期权被执行的概率
    \item $N(d_1)$ 是看涨期权价格对资产价格的导数,它反映了很短时间内期权价格变动预期标的资产价格变动的比率
    \item 资产的价格波动率用于度量资产所提供收益的不确定性,人们经常采用历史数据和隐含波动率来估计
\end{enumerate}
期权模型:希腊字母

$\Delta$ 是用来衡量标的资产价格变动对期权理论价格的影响程度,可以理解为期权对标的资产价格变动的敏感性

看涨期权:
$$\Delta=\frac{\partial C}{\partial S}=N(d_1)$$

看跌期权:
$$\Delta=\frac{\partial P}{\partial S}=N(d_1)-1$$

性质:随着到期日临近,看涨期权和看跌期权的 $\Delta$ 收敛情况如下:
\begin{itemize}
    \item 看涨期权
    \begin{itemize}
        \item 实值期权(标的价格大于行权价),收敛于1
        \item 平值期权(标的价格等于行权价),收敛于.5
        \item 虚值期权(标的价格小于行权价),收敛于0
    \end{itemize}
    \item 看跌期权
    \begin{itemize}
        \item 实值期权(标的价格小于行权价),收敛于-1
        \item 平值期权(标的价格等于行权价),收敛于-.5
        \item 虚值期权(标的价格大于行权价),收敛于0
    \end{itemize}
\end{itemize}

$\Delta$ 策略
\begin{itemize}
    \item $\Delta$ 对冲策略:利用期权价格对标的资产价格变动的敏感度为 $\Delta$,按照 1 单位资产和 $\Delta$ 单位期权做反向头寸来规避资产价格中价格波动风险;
    \item $\Delta$ 中性策略:如果 $\Delta$ 对冲策略能完全规避组合的价格波动风险,称该策略为 $\Delta$ 中性策略;
    \item 当标的资产价格大幅波动时,$\Delta$ 值也随之变化,静态的 $\Delta$ 对冲并不能完全规避风险,需要投资者不断依据市场变化调整对冲头寸
\end{itemize}

$\Gamma$ 值衡量 $\Delta$ 值对标的资产的敏感度。$\Gamma$ 值较小时,$\Delta$ 对资产价格变动不敏感,投资者不必频繁调整头寸对冲资产价格变动风险。反之,投资者就需要频繁调整头寸。

看涨期权:
$$\Gamma_c=\frac{\partial^2 C}{\partial S^2}=\frac{N(d_1)}{S\sigma\sqrt{T}}$$

看跌期权:
$$\Gamma=\frac{\partial^2 P}{\partial S^2}=\frac{N(d_1)}{S\sigma\sqrt{T}}$$

性质:
\begin{itemize}
    \item 看涨期权和看跌期权的 $\Gamma$ 值均为正值;
    \item 深度实值和深度虚值的期权 $\Gamma$ 值均较小,只有当标的资产价格和执行价相近时,价格的波动都会导致 $\Delta$ 值的剧烈变动,因此平价期权的 $\Gamma$ 最大;
    \item 期权到期日临近,平价期权的 $\Gamma$ 值趋近无穷大;实值和虚值期权的 $\Gamma$ 值先变大后变小,随着接近到期收敛值 0;
    \item 波动率和 $\Gamma$ 最大值呈反比,波动率增加将使行权价附近的 $\Gamma$ 减小,远离行权价的 $\Gamma$ 增加;
\end{itemize}

$Vega$ 用来衡量期权价格对波动率的敏感性,该值越大,表明期权价格对波动率的变化越敏感。

看涨期权:
$$v_c=\frac{\partial C}{\partial \sigma}=S\times \sqrt{T}\times N(d_1)$$

看跌期权:
$$v_p=\frac{\partial P}{\partial \sigma}=S\times \sqrt{T}\times N(d_1)$$

期权的波动率敏感度公式:
$$\Delta C=v\Delta \sigma$$

性质:
\begin{itemize}
    \item 波动率与期权价格成正比;
    \item 平价期权对波动率变动最为敏感,深度实值和深度虚值期权中资产价格和执行价对 $d_1$ 起到决定性作用,波动率的影响被弱化;
\end{itemize}

$\Theta$ 用来度量期权价格对到期日变动敏感度

看涨期权:
$$\Theta_c=\frac{\partial C}{\partial t}=\frac{\sigma S}{2\sqrt{T}}N(d_1)-Ke^{-rt}rN(d_2)$$

看跌期权:
$$\Theta_p=\frac{\partial P}{\partial t}=\frac{\sigma S}{2\sqrt{T}}N(d_1)-Ke^{-rt}r[N(d_2)-1]$$

性质:
\begin{itemize}
    \item 看涨期权和看跌期权的 $\Theta$ 值通常是负的,表明期权的价值会随着到期日的临近而降低;
    \item 在行权价附近,$\Theta$ 的绝对值最大,即在行权价附近,到期时间变化对期权价值的影响最大;
    \item 随着期权接近到期,平价期权受到的影响越来越大,而非平价期权收到的影响越来越小:平价期权的 $\Theta$ 是单调递减至负无穷大;非平价期权的 $\Theta$ 将先变小后变大,随着接近到期收敛至 0;
\end{itemize}

$\rho$ 用来度量期权价格对利率变动敏感性

看涨期权

$$\rho_p=\frac{\partial C}{\partial r}=KTe^{-rt}N(d_2)$$

看跌期权

$$\rho_c=\frac{\partial C}{\partial r}=KTe^{-rt}[N(d_2)-1]$$

性质:
\begin{enumerate}
    \item 看涨期权的 $\rho$ 是正的,看跌期权的 $\rho$ 是负的;
    \item 随标的价格的变化,$\rho$ 随着标的证券价格单调递增
    \begin{itemize}
        \item 对于看涨期权,标的价格越高,利率对期权价值的影响越大
        \item 对于看跌期权,标的价格越低,利率对期权价值的影响越小
        \item 越是实值的期权,利率变化对期权价值的影响越大
        \item 越是虚值的期权,利率变化对期权价值的影响越小
    \end{itemize}
    \item $\rho$ 随着期权到期,单调收敛到 0,即期权越接近到期,利率变化对期权价值的影响越小;
\end{enumerate}

\begin{table}
    \begin{tabular}{ccc}
        \hline
        希腊字母&风险因素&量化公式\\
        \hline
        $\Delta$&标的资产价格变化&权利金变动值/标的价格变动值\\
        $\Gamma$&标的价格变动&$\Delta$变动值/标的价格变动值\\
        $Vega$&波动率变化&权利金变动值/波动率变动值\\
        $\Theta$&到期时间变化&权利金变动值/到期时间变动值\\
        $\rho$&利率变动&权利金变动值/利率变动值\\
        \hline
    \end{tabular}
\end{table}

\section{波动率}
某个变量的波动率 $\sigma$ 定义为这一变量在单位时间内连续复利回报率的标准差。当波动率被用于期权定价时,时间单位通常为一年,因此波动率就是一年连续复利回报率的标准差,但是当波动率用于风险控制时时间单位通常是一天,此时的波动率对应于每天连续复利回报率的标准差。

波动率分为两种:
\begin{itemize}
    \item 回望型波动率(backward looking),用历史数据算出来的波动率,是已经发生了的历史价格的波动;
    \item 前瞻波动率(forward looking),根据现在的期权价格,用 B-M-S 期权定价模型反推出来的波动率,是未来一个价格的波动率的预测,未必准确;
\end{itemize}
\subsection{回望型波动率}
回望型波动率,就是我们所说的历史波动率,它是根据历史数据计算的,在这里,我们要区分日波动率和年波动率的概念。定义以下符号:$S_i, i=0,1,\cdots,n$ 为第 $i$ 个时间区间结束时变量的价格,$\tau$ 为时间区间的长度,单位是年。

收益率 $u_i$ 的计算公式为:
\begin{equation}
    u_i=\ln\frac{S_i}{S_{i-1}}
\end{equation}

$u_i$ 的标准差 $s$ 通常估计为:
$$s=\sqrt{\frac{1}{n-1}\sum_{i=1}^{n}(u_i-\bar{u})}$$
或者
$$s=\sqrt{\frac{1}{n-1}\sum_{i=1}^{n}u_i^2-\frac{1}{n(n-1)}(\sum_{i=1}^{n}u_i)^2}$$

\subsection{隐含波动率}
隐含波动率是通过 B-M-S 公式反解出来的波动率,一般认为是对未来波动率的预期。

\section{回测}
回测是根据历史数据来验证交易的可行性和有效性的过程。做回测是希望可以用回测后的表现来评估未来实盘表现。
\end{document}