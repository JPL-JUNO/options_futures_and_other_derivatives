\documentclass{article}
\usepackage{tikz}
\usepackage{ctex}
\usepackage{geometry}
\begin{document}
\begin{figure}
    \centering
    \begin{tikzpicture}[scale=3]
        \draw[<->] (0, 2)--(0,0)--(5, 0);
        \draw[thick] (0, 0)--(3, 0)--(4.5, 1.5);
        \draw (0, 0.02) .. controls (3., 0.05) .. (4.5, 1.52);
        \draw (0, 0.1) .. controls (3., 0.25) .. (4.5, 1.62);
        \draw (0, 0.15) .. controls (3., 0.35) .. (4.5, 1.75);
        \draw[dotted] (3, 0)--(3,.58);
        \node[above left] at (4.5, 1.55) {\tiny 9 个月曲线};
        \node at (2.1, 0.25) {\tiny 6 个月曲线};
        \node at (1.5, 0.05) {\tiny 3 个月曲线};
        \filldraw (3,0) circle (1pt) node[below] {行权价};
        \node[below] at (1.3, 0) {内在价值};
        % \draw[->](55, -8)--(55, 5);
        % \draw (55, -7)--(60, -7)--(67, 0)--(71, 4);
        % \draw (55, -3)--(70, -3)--(73, 0)--(77, 4);
        % \node[above right] at (71, 4) {实值看涨期权};
        % \node[above right] at (77, 4) {虚值看涨期权};
        % \filldraw (55, 0) circle (2pt) node[below left] {$O$};
        % \foreach \x in {60, 65, 67, 70, 73}{
        %         \filldraw (\x, 0) circle (2pt) node[below right] {\tiny $(\x, 0)$};
        %     }
        % \foreach \y in {-3, -7}{
        %         \node[left] at (55, \y) {$\y$};
        %     }
        % \draw[dotted] (60, 0)--(60, -7);
        % \draw[dotted] (70, 0)--(70, -3);
        % \draw[dotted, red, thick] (68, 1)--(68, -3);
        % \filldraw[red] (65, 0) circle (2pt) node[above left] {现价};
        % \filldraw[red] (64, -3) circle (5pt) node[below right] {平衡点};
    \end{tikzpicture}
    \caption{随着到期日的临近,更低的曲线开始向内在价值线靠拢。期权价格开始等于其内在价值}
    \label{fig3.1}
\end{figure}
\end{document}