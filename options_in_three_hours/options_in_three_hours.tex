\documentclass{article}
\usepackage{ctex}
\usepackage[top=2.45cm, bottom=2.45cm]{geometry}
\title{3 小时快学期权}
\author{Stephen CUI}
\date{2023-10-10}
\begin{document}
\maketitle
\section{短剑式:卖方策略}
短仓(义务仓,short option)指卖出期权、收入权利金,并负有义务的一方。
\subsection{怎样卖期权}
\subsubsection*{六个注意事项}
\begin{enumerate}
    \item 止损第一
    \item 仓位不能过重
    \item 顺势而为
    \item 多卖虚值期权
    \item 多卖被高估的期权
    \item 盯市、盯市、再盯市
\end{enumerate}
\subsection{保证金和卖方风险}
\section{化剑式:保险策略}
持股票,买认沽,股价下跌包财富。

买认购,借股票,卖空风险全送掉。

% 借股票,买认购,卖空风险全送掉。
\subsection{什么是保险策略}
期权保险策略可以分为两种:一种是在持有标的的证券或买入标的证券的同时,买入相应数量的认沽期权,为标的证券提供价格下跌的保险;另一种是融券卖出标的证券时买入相应数量的认购期权,为标的证券提供价格上涨的保险。
\section{重剑式:增强收益策略}
备兑开仓策略是指持有现货,预期未来上涨可能性不大,就可以卖出认购期权构成备兑组合,以增强收益。用卖出认沽期权锁定买入价是指准备买入股票,但认为现有股价过高或预期未来可能下跌,就可以卖出认沽期权以锁定买入价格,并通过收取期权费来降低建仓成本。(\textbf{卖出一个认股期权,锁定价格(因为是想买股票嘛)然后期权金还可以用来降低成本。})
\subsection{什么是备兑开仓策略}
\subsection{备兑开仓三个注意事项}
\paragraph{合约选择}
深度实值的期权,到期时被行权的可能性大,期权的时间价值相对较小,备兑投资的收益也相对较小;深度虚值的权利金则较小。因而最好选择平值或轻度虚值的合约。
\paragraph{心理准备}
\paragraph{及时调整}
\subsection{买股票还是卖认沽}
卖出认沽期权策略的不足之处主要体现在当标的价格出现大幅下跌,将面临保证金不足的风险和较大的亏损。
\section{花剑式:组合策略}
\subsection{合成股票策略}
\subsection{跨市和勒式策略}
\paragraph{买入跨市策略}
当投资者预期市场会出现大幅变动,但又不能准确判断标的证券变动方向时,可以买入相同数量、相同行权价的同月认购和认沽期权来构建买入跨市策略,捕捉市场大幅变动的收益,该策略构建成不能有限,理论上获得的收益无限。


\section{}
\end{document}