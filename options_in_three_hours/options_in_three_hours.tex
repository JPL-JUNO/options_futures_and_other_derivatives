\documentclass{article}
\usepackage{ctex}
\begin{document}
\section{短剑式:卖方策略}
短仓(义务仓,short option)指卖出期权、收入权利金,并负有义务的一方。
\subsection{怎样卖期权}
\subsubsection*{六个注意事项}
\begin{enumerate}
    \item 止损第一
    \item 仓位不能过重
    \item 顺势而为
    \item 多卖虚值期权
    \item 多卖被高估的期权
    \item 盯市、盯市、再盯市
\end{enumerate}
\subsection{保证金和卖方风险}
\section{化剑式:保险策略}
持股票,买认沽,股价下跌包财富。

买认购,借股票,卖空风险全送掉。

% 借股票,买认购,卖空风险全送掉。
\subsection{什么是保险策略}
期权保险策略可以分为两种:一种是在持有标的的证券或买入标的证券的同时,买入相应数量的认沽期权,为标的证券提供价格下跌的保险;另一种是融券卖出标的证券时买入相应数量的认购期权,为标的证券提供价格上涨的保险。
\section{重剑式:增强收益策略}
备兑开仓策略是指持有现货,预期未来上涨可能性不大,就可以卖出认购期权构成备兑组合,以增强收益。用卖出认沽期权锁定买入价是指准备买入股票,但认为现有股价过高或预期未来可能下跌,就可以卖出认沽期权以锁定买入价格,并通过收取期权费来降低建仓成本。(\textbf{})
\section{}
\section{}
\end{document}