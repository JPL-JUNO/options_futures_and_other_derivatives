\chapter{卖出备兑看涨期权\label{ch02}}
\section{卖出备兑看涨期权的哲学}
\subsection{卖出备兑的类型}
虽然所有的卖出备兑都涉及就持有的股票卖出看涨期权,但有不同的术语来描写不同类型的卖出备兑。有两个术语用得最为广泛,它们包含了所有的卖出备兑,一个是虚值卖出备兑(out-of-the-money covered write),另一个是实值卖出备兑(in-the-money covered write)。很明显,这两个术语指的是在初始建立头寸的时候,该期权自身究竟是实值还是虚值的。有时,卖出备兑的分类依据于所涉及的股票的性质(低价卖出备兑、高收益卖出备兑等),但这些都只是这两大类的子集。

一般而言,相对于实值卖出备兑来说,虚值卖出备兑能提供更高的潜在收益,但对风险的保护程度则较低。根据在建立合约时该看涨期权实值或虚值的程度,投资者可以建立一个进攻型或防守型的卖出备兑头寸。实值卖出更多地是防守型卖出备兑头寸。

\begin{figure}
    \centering
    \begin{tikzpicture}[scale=.32]
        \draw[thick, <->] (30, 13)--(30, 0)--(60, 0);
        \draw[thick] (40, -5)--(55, 10);
        \draw (30, 8)--(40, 8)--(48, 0)--(50, -2);
        \draw (30, 1)--(50, 1)--(55, -4);
        \filldraw[black] (40, 8) circle (5pt) node[above]{(40, 8)};
        \filldraw[black] (45, 0) circle (5pt) node[below]{(45, 0)};
        \filldraw[black] (50, 1) circle (5pt) node[right]{(50, 1)};
        \filldraw[black] (30, 0) circle (5pt) node[below left]{(30, 0)};
        \draw[red, thick] (40, -4)--(50, 6)--(55, 6);
        \draw[orange, thick] (33, -4)--(40, 3)--(55, 3);
        \node[above] at (56, 3){\tiny 实值卖出备兑(防守型)};
        \node[above] at (56, 6){\tiny 虚值卖出备兑(进攻型)};
        \draw[dotted] (30, 6)--(50, 6);
        \filldraw (50, 6) circle (5pt) node[above]{(50, 6)};
        \draw[dotted] (30, 3)--(40, 3);
        \filldraw (40, 3) circle (5pt) node[below]{(40, 3)};
        \filldraw[orange] (37, 0) circle (5pt) node[below right]{(37, 0)};
        \filldraw[red] (44, 0) circle (5pt) node[above left]{(44, 0)};
    \end{tikzpicture}
    \caption{XYZ 普通股股票的售价是 45,有两个期权在考虑范围之内:售价为 8 的 XYZ 7 月 40 看涨期权以及售价为 1 的 XYZ 7 月 50 看涨期权。7 月 40 的实值卖出备兑在到期日可以提供8点,也就是 18\% 的保护,可以保护到价格为 37 的程度(盈亏平衡点)。虚值的 7 月 50 的卖出备兑对到期日的价格下行风险只提供了 1 点的保护。因此,同虚值卖出备兑相比,实值卖出备兑提供了更大的对价格下行风险的保护。这个结论并不仅适用于这个示例,它是普遍正确的。}
\end{figure}

投资者可以通过卖出虚值备兑看涨期权来构建一个更具有进攻性的头寸。在这种情况下,投资者对标的股票应当是看多的。如果投资者对股票的看法是中性的或者略微看空的,则进行实值卖出备兑就更为合适。如果投资者对所持有的股票确实看空,那么他应该卖出股票,而不是建立一个卖出备兑。

\section{计算投资收益}
在建立头寸之前,有三个关于卖出备兑的基本因素需要计算。第一个因素是\textbf{行权收益}(return if exercise)。这是在股票被交割的情况下卖出者会得到的投资收益。对虚值卖出备兑来说,要得到行权收益,股票的价格就必须上涨。然而,对实值卖出备兑来说,即使在期权到期时股价没有变化,仍然可以得到行权收益。因此,更有利的方法是计算\textbf{无变化时收益}(return if unchanged),即如果期权到期时标的股价不变时的收益。此时,投资者可以使用股价不变收益来对虚值和实值卖出备兑做更公平的比较,这时对股价变动就没有做过多的假设。备兑卖出者应当考虑的第三个重要的统计数据是在考虑了所有费用之后的下行盈亏平衡点(downside break-even point)。只要知道了这个盈亏平衡点,投资者就可以轻易地计算出他从卖出看涨期权中所得到的\textbf{下行保护}(downside protection)的比率。

\subsection{一毛钱会造成多大区别}
与卖出备兑潜在收益有关的另一个重要方面,当然是交易中所涉及的股票和期权的价格。你买入股票时多付几分钱,或者卖出看涨期权时少收入一两毛钱,看上去似乎无关紧要,但是,即使是一个相对很小的数目也有可能导致潜在收益的数量发生令人惊讶的变化。虽然所有的卖出备兑都受此影响,但对实值期权的卖出者来说尤其如此。

\subsection{策略的重要性}
某个卖出备兑的策略家所要达到的结果,是在可接受的收益与下行保护之间寻找一种平衡。在这两者之中,如果有哪一个不能满足他的要求,他就不应当建立这样的策略。如果他对某只股票是看空的,他也不会建立这样的策略。从这点出发,卖出备兑所达到的结果就是某个保守型的卖出备兑程序所要实现的目的:为某个波动较小的投资组合增加收益,提供保护,降低组合波动率。
\section{在卖出备兑中对收益和保护分散化}
\subsection{分散化的基本方法}
为了尽可能达到分散化的目的,卖出者通常可以对他一半的头寸就一只股票卖出虚值的看涨期权,再对另一半就同一股票卖出实值的看涨期权。对下面这样的股票来说,这种方法特别具有吸引力:这种股票的虚值看涨期权看上去不能提供足够的下行保护,与此同时,它的实值看涨期权又提供不了足够的收益。通过同时卖出这两种期权,这个卖出者就有可能得到他所寻求的收益和保护的分散化。

\begin{figure}
    \centering
    \begin{tikzpicture}[scale=.6]
        \draw[<->] (30, 8)--(30, 0)--(50, 0);
        \draw (39, -3)--(42, 0)--(48, 6);
        \draw[dotted] (30, 4)--(40, 4)--(47, -3);
        \draw[dotted] (30, 2)--(45, 2)--(49, -2);
        \draw (35, -3)--(40, 2)--(49, 2);
        \draw (37, -3)--(45, 5)--(49, 5);
        \foreach \x in {38, 40, 42, 45}{
                \filldraw  (\x, 0) circle (2pt) node[below]{\tiny $(\x, 0)$};
            }
        \foreach \y in {2, 4}{
                \node[left] at (30, \y) {\y};
            }
        \draw[dotted] (40, 0)--(40, 4);
        \draw[dotted] (45, 0)--(45, 5);
        \draw[red, thick] (36, -3)--(40, 1)--(45, 3.5)--(49, 3.5);
        \node[above right] at (45, 2) {\tiny 卖出实值};
        \node[above right] at (45,3.5) {\tiny 卖出组合};
        \node[above right] at (45,5) {\tiny 卖出虚值};
    \end{tikzpicture}
    \caption{卖出组合同卖出实值和虚值相比}
    \label{fig2.2}
\end{figure}

\section{后续行动}
\textbf{后续行动}(follow-up action)可以分为三大类:
\begin{enumerate}
    \item 如果股价下跌而采取的保护性行动;
    \item 如果股价上涨而采取的进攻性行动;
    \item 如果实值看涨期权时间价值消失而采取的避免指派行动。
\end{enumerate}
有的时候,投资者在到期日之前决定将整个头寸平仓,或者让股票被指派行权。这些情况我们也会讨论。
\subsection{如果股价下跌而采取的保护性行动}
如果某个投资者面对标的股价相对显著的下跌而不采取保护性行动,就有可能遭受大笔的亏损。因为卖出备兑是一个潜在盈利有限的策略,投资者应当限制亏损,不然一次亏损可以抵消掉好几次盈利。在市场下跌中最简单的后续行动,就是直接将这个头寸平仓。如果股价下跌了一定百分点,或者是跌破了某个技术支撑(support)位,就应当采取这样的行动。不幸的是,这种防守行为有可能被证明是一种不高明的方式。投资者往往会继续卖出时间价格更高的期权,来获得额外的时间价值。

后续行动通常采取的形式是买回初始卖出的看涨期权,然后再卖出另一个行权价或到期日不同的看涨期权来替代。这种类型的调整叫作挪仓行动(rolling action)。当标的股价下跌时,投资者通常买回初始的看涨期权(由于标的股价下跌了,从而获得一定的盈利),然后再以较低的行权价卖出另一种看涨期权。这就是所谓的\textbf{向下挪仓}(rolling down),因为新的期权的行权价较低。

\begin{figure}
    \centering
    \begin{tikzpicture}[scale=.45]
        \draw[<->] (40, 8)--(40, 0)--(60, 0);
        \draw (49, -2)--(51, 0)--(57, 6);
        \draw[dotted] (40, 6)--(50, 6)--(58, -2);
        \foreach \y in {6, 3}{
                \node[left] at (40, \y) {\y};
            }
        \foreach \x in {42, 45, 51, 56}{
                \filldraw (\x, 0) circle (3pt) node[below  right]{\tiny $(\x, 0)$};
            }
        \draw (43,-2)--(45, 0)--(50, 5)--(60, 5);
        \draw[dotted] (50, 0)--(50, 6);
        \node[above right] at (58, 5) {\tiny 初始头寸};
        \draw[thick, red] (41, -1)--(45, 3)--(60, 3);
        \node[above right] at (58, 3) {\tiny 向下挪仓后的头寸};
        \filldraw[blue] (48, 3) circle (3pt) node[above left]{\tiny 表现相等};
    \end{tikzpicture}
    \caption{比较:初始头寸同向下挪仓后的头寸}
    \label{fig2.3}
\end{figure}
请注意,挪仓后头寸的最大潜在盈利要比初始的头寸小。向下挪仓一般会降低卖出备兑的最大潜在盈利。不过,当一只股票在下跌时,最大盈利也许是次要的考虑。在这样的情况下,额外的下行保护常常是更紧迫的需要。

向下挪仓不能带来好处的唯一可能原因就是股票大幅反弹,也就是说,价格先是下跌然后上涨。在选择向下挪仓时,挪到什么位置很重要,因为挪得过早或是挪到不适当的价格都会限制收益。在选择挪仓价格时,使用股票的技术支撑位常常很有用。一般而言,如果投资者在技术支撑位被突破之后再向下挪仓,就可以减小陷入股价反弹这种处境的可能。

在现实实践中可能会出现更复杂的情况,例如标的股票的价格突然出现相对陡峭的跌幅。这可能会给卖出者带来所谓的锁定亏损(locked-in loss)。简单地说,这就意味着对该卖出者来说,不存在这样的期权:它们可以让他向下挪仓,从而得到足够的权利金,如果股票在到期时被指派行权,就能兑现由此发生的盈利。这类情况通常更多地发生在价格较低的股票上,这些股票的期权的行权价离市场价相对较远。相对于卖出实值来说,卖出虚值更有可能会遇到这样的问题。尽管从情感上来说,此时向下挪仓无法产生盈利(至少在一段时间内),不是一个令人满意的投资头寸。但是,这样做至少为股价下跌提供了尽可能的保护,因此它仍是有益的。

\subsection{一个替代向下挪仓的办法}
备兑卖出者可以采取另一种方法来得到对市场下跌的额外保护,并且不一定需要锁定亏损。简单地说,该卖出者可以只将他的一部分卖出备兑头寸进行挪仓。

对那些想要向下挪仓但又不想锁定亏损的备兑卖出者,或者是认为股票在到期日之前会有一定反弹的人来说,应当考虑只将他的部分头寸向下挪仓。如果股票继续下跌,出现明显不能再强有力地反弹回初始的行权价的迹象时,那么,投资者可以再把剩余部分头寸也向下挪仓。

\subsection{向下挪仓时利用不同的到期月系列}
如果卖出者因为诸如标的股价突然下跌这种不受他控制的原因而不得不向下挪仓并且锁定亏损的话,事实上他可以挪仓到近期月的合约。这样他就可以在尽可能短的时间里拿回从短期看涨期权中得到的时间价值。
\subsection{股票上涨时采取的行动}
一种让备兑卖出者高兴的情况是,标的股票的价格在建立了卖出备兑头寸之后上涨。如果发生这样的事,一般而言有好几种选择。卖出者可以决定什么都不做,让股票指派行权,从而得到建立这个头寸时所希望的收益。另一方面,如果标的股票价格上涨得非常快,卖出的看涨期权达到了持平价,卖出者也可以提前将头寸平仓,或者将看涨期权向上挪仓。下面我们对这两种情况各作一些讨论。

\begin{figure}
    \centering
    \begin{tikzpicture}[scale=.5]
        \draw[<->] (40, 15)--(40, 0)--(65, 0);
        \draw (48, -2)--(60, 10);
        \draw[dotted] (40, 6)--(50, 6)--(58, -2);
        \draw (42, -2)--(50, 6)--(60, 6);
        \draw[dotted] (40, 12)--(60, 12);
        \draw[dotted] (60, 12)--(60, 0);
        \foreach \y in {6, 12}{
                \node[left] at (40, \y) {\y};
            }
        \foreach \x in {44, 48, 50, 56}{
                \filldraw (\x, 0) circle (3pt) node[below  right]{\tiny $(\x, 0)$};
            }
        \draw[dotted] (50, 0)--(50, 6);
        \node[above right] at (53, 6) {\tiny 初始头寸};
        \draw[red, thick] (46, -2)--(60, 12)--(63, 12);
        \node[above right] at (60, 12) {\tiny 向上挪仓后的头寸};
    \end{tikzpicture}
    \caption{比较:初始头寸与向上挪仓后的头寸}
    \label{fig2.4}
\end{figure}

总的来说,向上挪仓增加了投资者盈利的潜在可能,但是,也增加了一旦股价朝反向运动就必须面临的风险暴露。因此,它所引进的就不仅是增进收益的可能性,还有风险因素。一般而言,如果不能承受股价至少 10\% 的回调,那就不应当向上挪仓。卖出备兑的初始目的在建立头寸时就已经确定。当股票上升,这些目的已经达到时,卖出者就应当仔细考虑,是否要把盈利置身于风险中。
\section{局部抽身策略}
当股价上涨,股票有被指派的风险时,就可以应用该策略。如果备兑卖出者希望被指派,那就没有什么问题。但在很多情况下,卖出者都希望买回看涨期权和向上挪仓。在这个策略中,部分标的股票被卖出了,收回的资金可用来买回备兑看涨期权。如果愿意的话,此时还可以卖出更高行权价的新看涨期权。

抽身策略有一些好的特征,其中最大的一点是,投资者通过卖出少量的股票,就让自己不用再担心实值备兑看涨期权的指派问题。该策略可以在备兑看涨期权变得非常深度实值之前就使用。如果某个投资者在使用该策略之前,期权已经变得非常深度实值了,那就必须卖出更多的股票,来让其从卖出备兑交易中抽身出来。
\subsection{什么时候让股票被指派行权}
如果满足下面的两个条件,让股票被指派行权的选择一般而言就是最聪明的策略:
\begin{enumerate}
    \item 向前挪仓只能提供很有限的收益;
    \item 向上或向前挪仓会显著地提高盈亏平衡点,使得该头寸相对来说失去了股价下跌时所需的保护。
\end{enumerate}

\section{对卖出备兑看涨期权的总结}

卖出备兑看涨期权是一个有活力的策略,因为它减少了持有股票所带来的风险,同时也减小了短期市场运动给某个投资者的投资组合所带来的波动性。不过,投资者应当知道,卖出备兑看涨期权的表现有可能会比只持有股票差,因为股票有可能会大幅上涨,而卖出备兑则对潜在盈利的上限进行了限制。选择什么样的看涨期权来卖出,取决于这个卖出备兑是趋于激进或趋于保守的。相对于卖出虚值看涨期权,卖出实值看涨期权从策略上来说要更为保守一些,因为它得到了更大份额的市场下行保护。卖出备兑看涨期权的总收益的概念,是想在各种来源的收入(期权权利金、股票持有权和股息)和市场下行保护之间维持最大的平衡。为了实现这个平衡,通常在股价接近行权价,或稍微实值,或稍微虚值时卖出备兑看涨期权。

在建立头寸之前,卖出者应当对各种收益进行计算:行权收益、股价不变收益以及盈亏平衡点。要对各种卖出作真正的比较,就必须将收益年化,而且,在计算中,必须包括所有的手续费和股息。如果持有较大的标的股票头寸(500 股或 1000 股),那么收益就会增加。同时,通过使用经纪公司的设施来产生所谓的“净”执行,也就是以某个特定的净价格差来买入股票和卖出看涨期权,那投资者就可以得到较好的执行结果,并能在长期中获得更高的收益。

应当在对可能有的收益和市场下行保护进行比较之后,再决定卖出哪个看涨期权。投资者有时可以卖出一部分虚值看涨期权和另一部分实值看涨期权,以实现收入和保护之间的平衡(\autoref{fig2.2})。最后,如果投资者对某只股票是看空的,他就不应当就其卖出看涨期权。卖出者对标的股票应当是略为看多的,或者至少是中性的。

后续行动与初始头寸的选择一样重要。如果标的股价下跌,投资者通过向下挪仓(\autoref{fig2.3}),可以增加下行的保护和当前的收入。如果某个投资者不愿意过多地限制市场上升时的潜在盈利,他可以考虑只将一部分备兑看涨期权头寸向下挪仓。在卖出的看涨期权到期时,如果股票相对接近于初始的行权价,投资者就应当考虑向前挪仓到一个更远的到期月。这样做能够获得更高的持续收益,因为投资者不必由于股票被指派行权而支付额外的股票手续费。在标的股票价格上升时,也可以采取一些进攻型的后续行动:向上挪仓(\autoref{fig2.4})到一个更高的价位。这个行动可以增加最大潜在盈利,但是,如果标的股票价格有显著的下跌,就会使得头寸面临亏损。如果股价高于行权价,而且其他证券里有更好的收益机会,投资者就不应当采取任何后续行动,而是让他的股票被指派。