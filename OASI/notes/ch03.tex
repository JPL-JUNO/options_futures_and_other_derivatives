\chapter{买入看涨期权}
买入看涨期权策略的成功,主要取决于投资者是否能选到会上涨的股票,以及能把握住这个机会。因此,虽然我们把买入看涨期权也称为策略,但策略这个词用在这里,与我们所讨论的大部分其他策略,涵义并不相同。设计其他大部分策略的目的,是要去掉股票选择方面的某些精确性,使得投资者能够保持中性,或至少能有一定的犯错余地,同时还可以盈利。
\section{为什么要买入看涨期权}
买入看涨期权的最大吸引力是它们能为投机者提供很大的杠杆。

有的投资者买入看涨期权的目的是想在把风险控制在一个固定金额的情况下,为其组合增加某种上行潜在收益。

有的投资者是因为另外一个理由而买入看涨期权,就是能够在不错过市场的情况下按合理的价格买入股票。

\begin{figure}
    \centering
    \begin{tikzpicture}[scale=.5]
        \draw[->] (55, 0)--(80, 0);
        \draw[->](55, -8)--(55, 5);
        \draw (55, -7)--(60, -7)--(67, 0)--(71, 4);
        \draw (55, -3)--(70, -3)--(73, 0)--(77, 4);
        \node[above right] at (71, 4) {实值看涨期权};
        \node[above right] at (77, 4) {虚值看涨期权};
        \filldraw (55, 0) circle (2pt) node[below left] {$O$};
        \foreach \x in {60, 65, 67, 70, 73}{
                \filldraw (\x, 0) circle (2pt) node[below right] {\tiny $(\x, 0)$};
            }
        \foreach \y in {-3, -7}{
                \node[left] at (55, \y) {$\y$};
            }
        \draw[dotted] (60, 0)--(60, -7);
        \draw[dotted] (70, 0)--(70, -3);
        \draw[dotted, red, thick] (68, 1)--(68, -3);
        \filldraw[red] (65, 0) circle (2pt) node[above left] {现价};
        \filldraw[red] (64, -3) circle (5pt) node[below right] {平衡点};
    \end{tikzpicture}
    \caption{相对于购买实值看涨期权,购买虚值看涨期权的潜在风险和潜在收益都更大。就百分比来看,当股票上升幅度不大时,实值看涨期权能够提供更好的收益,而虚值看涨期权则在股票价格上升幅度更大的情况下会表现更好。}
    \label{fig3.1}
\end{figure}
\subsection{时机的确定}
如果投资者相当肯定标的股票马上就会上涨,他就应当努力得到更多的收益,而不必那么担心风险。这就意味着应买入短期的、稍稍虚值的看涨期权。当然,这只是一般的规则。无论在什么情况下,投资者一般都不要买入离到期只有一个星期的虚值看涨期权。另一方面,如果投资者在时机把握上非常欠缺,他就应当买入长期的看涨期权,这样,如果他对时机的把握差错很大,可以降低他的风险。

许多情况下,投资者并不打算长期持有看涨期权,他只是想在标的股票的短期快速上涨中获利。在这种情况下,他应当买入相对短期的实值看涨期权。尽管这个看涨期权可能要比同一标的股票的虚值看涨期权更贵一些,但只要标的股票价格上涨,这个期权就几乎肯定会上涨。因此,这个短期的交易者就会盈利。

\subsection{delta}
\begin{figure}
    \centering
    \begin{tikzpicture}[scale=3]
        \draw[<->] (0, 2)--(0,0)--(5, 0);
        \draw[thick] (0, 0)--(3, 0)--(4.5, 1.5);
        \draw (0, 0.02) .. controls (3., 0.05) .. (4.5, 1.52);
        \draw (0, 0.1) .. controls (3., 0.25) .. (4.5, 1.62);
        \draw (0, 0.15) .. controls (3., 0.35) .. (4.5, 1.7);
        \draw[dotted] (3, 0)--(3,.58);
        \node[above left] at (4.5, 1.55) {\tiny 9 个月曲线};
        \node at (2.1, 0.25) {\tiny 6 个月曲线};
        \node at (1.5, 0.05) {\tiny 3 个月曲线};
        \filldraw (3,0) circle (1pt) node[below] {行权价};
        \node[below] at (1.3, 0) {内在价值};
    \end{tikzpicture}
    \caption{随着到期日的临近,更低的曲线开始向内在价值线靠拢。期权价格开始等于其内在价值}
    \label{fig3.2}
\end{figure}

当股票价格与看涨期权行权价相同时,时间价值最高;它是实值或虚值程度最浅的期权;期权的价格不是线性因时减值的,随着到期日的临近,时间价值消失的速度会更快。请注意,上面提到的所有事实都可以在\autoref{fig3.2}中观察到。与中间区域相比,这些曲线在两头时与“内在价值”线的距离更近,这就意味着时间价值在股票价格等于行权价时最大,在股票价格偏离行权价时最小,无论是向实值方向还是向虚值方向偏离。此外,3 个月期的期权曲线位于内在价值线和 9 个月期的期权曲线的中间,这意味着当期权是平值的或近值(near-the-money)时,它的因时减值率是呈非线性的。

如果某个日内交易者坚持要使用期权,那就应当买入短期实值期权,因为它的 delta 在所有期权中最高,最好 delta 接近 0.90 或者更高。这样的期权对标的物的小幅度运动也能有迅速的反应。

买家应当熟悉看涨期权的另一个特性,就是期权的\textbf{对冲比率}(delta)。简单地说,期权的 delta 就是当标的股票运动1点时,看涨期权的价格会上涨或下跌的数量。

对所有看涨期权的买家来说,delta 都是一则重要的信息,因为它可以告诉买家,标的股票的短期运动会带来多大的增值或减值。这个信息可以帮助看涨期权的买家决定买入什么样的看涨期权。
\section{买什么样的期权}
一般的规则是:策略的期限越短,用来交易这个策略的工具的 delta 就应当越高。
\subsection{日内交易}
现在日内交易变得越来越普遍了。根据统计资料,大多数做日内交易的人都是赔钱的。许多希望使用期权的日内交易者没有意识到,就日内交易而言,应当使用 delta 值最高的工具。这个工具就是 delta 为 1.0 的标的资产。
\subsection{短线交易}
假定某个投资者想通过使用某个策略来达到持有标的物 $1 \sim 2$ 个星期的目的,就像在日内交易的情况中一样,在这种情况里,他也需要较高的 delta 值。在这种情况里应当使用高 delta 值期权的理由之一是,投资者对日内交易的时机或者特短期的交易方法相当有把握。如果选股的方法在把握时机方面有高度的精确性,那么,就应当使用高 delta 值的期权。
\subsection{中线交易}
随着投资者交易策略周期的拉长,使用较低 delta 值的期权就更为合适。这通常意味着在选择时机方面不需要再那么精确。这些策略的投资者应当使用 delta 较小的期权。想在几个星期、几个月甚至更长时间里持有头寸的投资者,应当知道在这段时间里很可能会出现大的价格变化,那他就需要这样的期权来限制风险。因此,在这种情况下,他就应当使用平值期权。
\subsection{长线交易}
如果投资者的策略是个长期策略,他就应当考虑 delta 更低的期权。这样的策略在把握时机的能力方面一般都很模糊。一般我们不提倡买入虚值期权,但是,对期限很长的策略来说,投资者可以考虑选择略为虚值的期权,或者至少是存续期较长的平值期权。无论是哪种情况,相对于我们在上面为其他策略推荐的期权来说,这里使用的期权的 delta 值都会比较小。此外,对于这一类的股票策略,长期期权可能是合适的。
