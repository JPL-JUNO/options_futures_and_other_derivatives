\chapter{定义\label{ch01}}
\section{基本定义}
股票期权(stock option)是一种在未来某一有限时间段内以某一特定价格买入或卖出某一特定股票的权利。这里所说的股票被称为标的证券(underlying security)。看涨期权(call)赋予拥有者(或持有者)买入标的证券的权利,而看跌期权(put)则赋予持有者卖出标的证券的权利。该股票可能会被买入或被卖出的价格被称为行权价(exercise price),也叫执行价(striking price)(在场内期权市场里,“行权价”和“执行价”是同义词)。股票期权所赋予的买入或卖出的权利只在有限的时间段内有效,因此,每个期权都有一个到期日(expiration date)。这里所说的期权始终是指场内期权(listed option),即存在二级市场,在期权交易所交易的期权。除特别提到之外,场外期权(over-the-counter options,OTC)不包括在我们的讨论范围之内。
\subsection{如何描述期权}
任何期权合约都有以下四个特征:
\begin{enumerate}
    \item 类型(看涨期权或看跌期权);
    \item 标的股票的名字;
    \item 到期日;
    \item 行权价。
\end{enumerate}
\subsection{期权的价值}
期权是一种\textbf{“消耗性”资产},也就是说,它的初始价值会随着时间的流逝而降低(或者说“消耗掉”)。有时它甚至会到期一文不值,持有者可能必须在其到期前行权,以挽回一些损失。当然,持有者也可以在到期前将其在场内期权市场中卖掉。
\subsection{期权价格与股票价格之间的关系}
实值(in-the-money)和虚值(out-of-the-money)。有特定的术语来描述股票价格与期权行权价之间的关系。如果股票价格低于看涨期权的行权价,则该看涨期权被称为虚值。如果股票价格高于看涨期权的行权价,则该看涨期权被称为实值(看跌期权则刚好相反,后文会进行讨论)。

实值看涨期权的内在价值(intrinsic value)等于股票价格超出行权价的那部分金额。如果该看涨期权是虚值的,则它的内在价值为零。卖出期权的价格一般被称为权利金(premium)。权利金与时间价值权利金[timevalue premium,简称为时间价值(time premium)]有明显的区别。时间价值等于该期权权利金超过其内在价值的那部分金额。我们可以用下面的公式来快速计算出实值看涨期权的时间价值:
$$\text{实值看涨期权的时间价值}=\text{看涨期权的价格}+\text{行权价}-\text{股票的价格}$$

持平(parity)。按其内在价值进行交易的期权被称为与标的证券持平的期权。因此,如果 XYZ 为 48,XYZ 7 月 45 看涨期权的售价为 3,这个看涨期权就是持平的(at parity)。期权卖出方特别感兴趣的一个常用方法(我们在后文会见到),就是通过指出一个期权与其持平状态普通股之间的距离来说明它的价格。
\section{影响期权价格的因素}
期权的价格与标的股票和该期权条款的性质相关。影响期权价格的主要可量化因素包括:
\begin{enumerate}
    \item 标的股票的价格;
    \item 期权自身的行权价;
    \item 该期权的剩余存续期;
    \item 标的股票的波动率;
    \item 当前无风险利率(例如 90 天政府债券利率);
    \item 标的股票的股息率。
\end{enumerate}
前四项是期权价格的主要决定因素,后两项一般没有那么重要,不过,对于高股息股票来说,股息率可能会有显著的影响。
